local Rayfield = loadstring(game:HttpGet('https://sirius.menu/rayfield'))()
 
local Window = Rayfield:CreateWindow({
   Name = "BluuGui",
   Icon = 0, -- Icon in Topbar. Can use Lucide Icons (string) or Roblox Image (number). 0 to use no icon (default).
   LoadingTitle = "Coolest Gui ever",
   LoadingSubtitle = "by 118o8",
   ShowText = "BluuGui ", -- for mobile users to unhide rayfield, change if you'd like
   Theme = "Ocean", -- Check https://[Log in to view URL]
 
   ToggleUIKeybind = "B", -- The keybind to toggle the UI visibility (string like "K" or Enum.KeyCode)
 
   DisableRayfieldPrompts = false,
   DisableBuildWarnings = false, -- Prevents Rayfield from warning when the script has a version mismatch with the interface
 
   ConfigurationSaving = {
      Enabled = true,
      FolderName = nil, -- Create a custom folder for your hub/game
      FileName = "Big Hub"
   },
 
   Discord = {
      Enabled = false, -- Prompt the user to join your Discord server if their executor supports it
      Invite = "noinvitelink", -- The Discord invite code, do not include discord.gg/. E.g. discord.gg/ ABCD would be ABCD
      RememberJoins = true -- Set this to false to make them join the discord every time they load it up
   },
 
   KeySystem = true, -- Set this to true to use our key system
   KeySettings = {
      Title = "Red guy Sucks?",
      Subtitle = "Say True",
      Note = "Tip: say YesHeSucks", -- Use this to tell the user how to get a key
      FileName = "Question", -- It is recommended to use something unique as other scripts using Rayfield may overwrite your key file
      SaveKey = true, -- The user's key will be saved, but if you change the key, they will be unable to use your script
      GrabKeyFromSite = true, -- If this is true, set Key below to the RAW site you would like Rayfield to get the key from
      Key = {"https://pastebin.com/raw/rpqKvd9K"} -- List of keys that will be accepted by the system, can be RAW file links (pastebin, github etc) or simple strings ("hello","key22")
   }
})
 
local MainTab = Window:CreateTab("Player", 4483362458) -- Title, Image
local MainSection = MainTab:CreateSection("Movility")
 
Rayfield:Notify({
   Title = "Notification",
   Content = "scipt have been executed correctly",
   Duration = 3,
   Image = 4483362458,
})

-- 🎵 Script de sonido al morir integrado en la GUI
local Players = game:GetService("Players")
local LocalPlayer = Players.LocalPlayer

local SoundIds = {
    72861611511867,
    128811607334302,
    93997080329253
}

-- Función para reproducir sonido aleatorio
local function PlayDeathSound()
    local humanoid = LocalPlayer.Character and LocalPlayer.Character:FindFirstChildOfClass("Humanoid")
    if humanoid then
        humanoid.Died:Connect(function()
            local randomId = SoundIds[math.random(1, #SoundIds)]
            local sound = Instance.new("Sound")
            sound.SoundId = "rbxassetid://" .. randomId
            sound.Volume = 2 -- 🔊 Doble volumen
            sound.Parent = workspace
            sound:Play()
            game:GetService("Debris"):AddItem(sound, 5)
        end)
    end
end

-- Conectar al cargar personaje
LocalPlayer.CharacterAdded:Connect(function()
    task.wait(1) -- espera un poco a que cargue
    PlayDeathSound()
end)

-- Si ya tiene personaje
if LocalPlayer.Character then
    PlayDeathSound()
end
 
local Button = MainTab:CreateButton({
   Name = "Inf Jump",
   Callback = function()
local Player = game:GetService'Players'.LocalPlayer;
local UIS = game:GetService'UserInputService';
 
_G.JumpHeight = 50;
 
function Action(Object, Function) if Object ~= nil then Function(Object); end end
 
UIS.InputBegan:connect(function(UserInput)
    if UserInput.UserInputType == Enum.UserInputType.Keyboard and UserInput.KeyCode == Enum.KeyCode.Space then
        Action(Player.Character.Humanoid, function(self)
            if self:GetState() == Enum.HumanoidStateType.Jumping or self:GetState() == Enum.HumanoidStateType.Freefall then
                Action(self.Parent.HumanoidRootPart, function(self)
                    self.Velocity = Vector3.new(0, _G.JumpHeight, 0);
                end)
            end
        end)
    end
end)
   end,
})
 
        local Slider = MainTab:CreateSlider({
   Name = "WalkSpeed",
   Range = {16, 100},
   Increment = 1,
   Suffix = "Speed",
   CurrentValue = 16,
   Flag = "Slider1", -- A flag is the identifier for the configuration file, make sure every element has a different flag if you're using configuration saving to ensure no overlaps
   Callback = function(Value)
        game.Players.LocalPlayer.Character.Humanoid.WalkSpeed = (Value)
   end,
})

		local Button = MainTab:CreateButton({
   Name = "BluVelosity (q)",
   Callback = function()
   		loadstring(game:HttpGet("https://pastebin.com/raw/VDxFA1Ze"))()
   end,
})
 
        local Button = MainTab:CreateButton({
   Name = "Fly (Press F)",
   Callback = function()
   -- LocalScript dentro de StarterPlayerScripts
local Players = game:GetService("Players")
local RunService = game:GetService("RunService")
local UserInputService = game:GetService("UserInputService")
 
local player = Players.LocalPlayer
local Fly = false -- Estado inicial
local FlySpeed = 50 -- Velocidad de vuelo
 
-- Tecla F para activar/desactivar
UserInputService.InputBegan:Connect(function(input, gameProcessed)
	if gameProcessed then return end
	if input.KeyCode == Enum.KeyCode.F then
		Fly = not Fly
	end
end)
 
-- Loop de vuelo
RunService.RenderStepped:Connect(function()
	if Fly then
		local char = player.Character
		if char and char:FindFirstChild("HumanoidRootPart") then
			local hrp = char.HumanoidRootPart
			local move = Vector3.new(0,0,0)
			local cam = workspace.CurrentCamera.CFrame
 
			-- Movimiento WASD + SPACE/SHIFT
			if UserInputService:IsKeyDown(Enum.KeyCode.W) then move = move + cam.LookVector end
			if UserInputService:IsKeyDown(Enum.KeyCode.S) then move = move - cam.LookVector end
			if UserInputService:IsKeyDown(Enum.KeyCode.A) then move = move - cam.RightVector end
			if UserInputService:IsKeyDown(Enum.KeyCode.D) then move = move + cam.RightVector end
			if UserInputService:IsKeyDown(Enum.KeyCode.Space) then move = move + Vector3.new(0,1,0) end
			if UserInputService:IsKeyDown(Enum.KeyCode.LeftShift) then move = move - Vector3.new(0,1,0) end
 
			if move.Magnitude > 0 then
				hrp.Velocity = move.Unit * FlySpeed
			else
				hrp.Velocity = Vector3.new(0,0,0)
			end
		end
	end
end)
 
   end,
})
 
        local Slider = MainTab:CreateSlider({
   Name = "JumpPower",
   Range = {0, 100},
   Increment = 1,
   Suffix = "Stuts",
   CurrentValue = 50,
   Flag = "Slider2", -- A flag is the identifier for the configuration file, make sure every element has a different flag if you're using configuration saving to ensure no overlaps
   Callback = function(Value)
          game.Players.LocalPlayer.Character.Humanoid.JumpPower= (Value)
   end,
})
 
        local Button = MainTab:CreateButton({
   Name = "Tp on click (Press X to tp)",
   Callback = function()
  -- LocalScript dentro de StarterPlayerScripts
local Players = game:GetService("Players")
local player = Players.LocalPlayer
local mouse = player:GetMouse()
 
-- Tecla X para teleport
local TPKey = Enum.KeyCode.X
 
local UserInputService = game:GetService("UserInputService")
 
UserInputService.InputBegan:Connect(function(input, gameProcessed)
	if gameProcessed then return end
	if input.KeyCode == TPKey then
		local char = player.Character
		local hrp = char and char:FindFirstChild("HumanoidRootPart")
		if hrp and mouse.Target then
			-- Teleportar ligeramente arriba del punto para no caer dentro del suelo
			hrp.CFrame = CFrame.new(mouse.Hit.Position + Vector3.new(0,3,0))
		end
	end
end)
 
   end,
})
 
        local Misc = Window:CreateTab("Misc", 4483362458) -- Title, Image
        local Section = Misc:CreateSection("SelfDefense")
 
        local Button = Misc:CreateButton({
   Name = "Loop Health Safe (SemiGod mode)",
   Callback = function()
                    local player = game.Players.LocalPlayer
local character = player.Character or player.CharacterAdded:Wait()
local humanoid = character:WaitForChild("Humanoid")

local godmodeEnabled = false

-- Toggle Godmode with G key
game:GetService("UserInputService").InputBegan:Connect(function(input, gameProcessed)
    if gameProcessed then return end
    if input.KeyCode == Enum.KeyCode.G then
        godmodeEnabled = not godmodeEnabled
        if godmodeEnabled then
            print("Godmode: ON")
            -- Auto heal loop
            task.spawn(function()
                while godmodeEnabled and humanoid and humanoid.Health > 0 do
                    if humanoid.Health < humanoid.MaxHealth then
                        humanoid.Health = humanoid.MaxHealth
                    end
                    task.wait(0.2)
                end
            end)
        else
            print("Godmode: OFF")
        end
    end
end)

-- Update references on respawn
player.CharacterAdded:Connect(function(char)
    character = char
    humanoid = character:WaitForChild("Humanoid")
end)

    end,
})

local Button = Misc:CreateButton({
   Name = "ForceField",
   Callback = function()
   local player = game.Players.LocalPlayer
local character = player.Character or player.CharacterAdded:Wait()

local forceFieldEnabled = false
local forceFieldInstance = nil

local function toggleForceField()
    local char = player.Character
    if forceFieldEnabled then
        -- Remove existing ForceField
        if char and char:FindFirstChildOfClass("ForceField") then
            char:FindFirstChildOfClass("ForceField"):Destroy()
        end
        forceFieldInstance = nil
        forceFieldEnabled = false
        print("ForceField: OFF")
    else
        -- Add ForceField
        if char and not char:FindFirstChildOfClass("ForceField") then
            forceFieldInstance = Instance.new("ForceField")
            forceFieldInstance.Parent = char
        end
        forceFieldEnabled = true
        print("ForceField: ON")
    end
end

game:GetService("UserInputService").InputBegan:Connect(function(input, gameProcessed)
    if gameProcessed then return end
    if input.KeyCode == Enum.KeyCode.F then
        toggleForceField()
    end
end)

-- Reapply ForceField on respawn if enabled
player.CharacterAdded:Connect(function(char)
    character = char
    if forceFieldEnabled then
        if not char:FindFirstChildOfClass("ForceField") then
            forceFieldInstance = Instance.new("ForceField")
            forceFieldInstance.Parent = char
        end
    end
end)

   end,
})
        
       -- NOCLIP
local noclipEnabled = false
local noclipConnection
local function toggleNoclip(state)
    noclipEnabled = state
    if state then
        noclipConnection = game:GetService("RunService").Stepped:Connect(function()
            for _, part in ipairs(game.Players.LocalPlayer.Character:GetDescendants()) do
                if part:IsA("BasePart") then
                    part.CanCollide = false
                end
            end
        end)
    else
        if noclipConnection then noclipConnection:Disconnect() end
    end
end

MainTab:CreateToggle({
    Name = "Noclip",
    CurrentValue = false,
    Flag = "NoclipToggle",
    Callback = toggleNoclip
})
        -- FULL BRIGHT
local fullBrightEnabled = false
local function toggleFullBright(state)
    fullBrightEnabled = state
    if state then
        game.Lighting.Brightness = 2
        game.Lighting.ClockTime = 12
        game.Lighting.FogEnd = 1e6
        game.Lighting.GlobalShadows = false
        game.Lighting.OutdoorAmbient = Color3.fromRGB(128,128,128)
    else
        -- Reset Lighting (puedes ajustarlo)
        game.Lighting.Brightness = 1
        game.Lighting.ClockTime = 14
        game.Lighting.FogEnd = 1000
        game.Lighting.GlobalShadows = true
        game.Lighting.OutdoorAmbient = Color3.fromRGB(0,0,0)
    end
end

Misc:CreateToggle({
    Name = "Full Bright",
    CurrentValue = false,
    Flag = "FullBrightToggle",
    Callback = toggleFullBright
})
 
                local Button = Misc:CreateButton({
   Name = "Anti Kick (client Sided)",
   Callback = function()
           loadstring(game:HttpGet("https://raw.githubusercontent.com/Exunys/Anti-Kick/main/Anti-Kick.lua"))()
   end,
})
 
        local Button = Misc:CreateButton({
   Name = "Unban VC",
   Callback = function()
           voiceChatService = game:GetService("VoiceChatService")
voiceChatService:joinVoice()
   end,
})
 
        local Button = Misc:CreateButton({
   Name = "ESP",
   Callback = function()
           loadstring(game:HttpGet("https://raw.githubusercontent.com/AlexDevlpr/Simple-ESP-using-Highlight-Roblox-/main/Minified_ver.lua"))()
   end,
})
 
        local Button = Misc:CreateButton({
   Name = "FE Invisible",
   Callback = function()
 
loadstring(game:HttpGet('https://pastebin.com/raw/3Rnd9rHf'))()
   end,
})
 
-- Crear la tab Scripts
local ScriptsTab = Window:CreateTab("Scripts", 4483362458)
 
-- Crear sección Hubs dentro de Scripts
local HubsSection = ScriptsTab:CreateSection("Hubs")
 
-- Botones dentro de la sección Hubs
local Button = ScriptsTab:CreateButton({
    Name = "NameLess Admin",
    Callback = function()
        loadstring(game:HttpGet("https://github.com/FilteringEnabled/NamelessAdmin/blob/main/Source?raw=true"))()
    end,
})
 
local Button = ScriptsTab:CreateButton({
    Name = "Infinite Yield",
    Callback = function()
        loadstring(game:HttpGet("https://rawscripts.net/raw/Universal-Script-Infinite-Yield-43437"))()
    end,
})
 
local Button = ScriptsTab:CreateButton({
    Name = "GhostHub",
    Callback = function()
        loadstring(game:HttpGet("https://rawscripts.net/raw/Universal-Script-X-Ghost-Hub-X-7595"))()
    end,
})
 
local Button = ScriptsTab:CreateButton({
    Name = "Aimbot",
    Callback = function()
       loadstring(game:HttpGet("https://raw.githubusercontent.com/agreed69-scripts/open-src-scripts/refs/heads/main/Universal%20Aimbot.lua",true))()()
    end,
})
        local Utility = Misc:CreateSection("Utility")
 
        local Button = Misc:CreateButton({
   Name = "Check FPS and ping",
   Callback = function()
           local player = game.Players.LocalPlayer
local gui = player:WaitForChild("PlayerGui")
local StarterGui = game:GetService("StarterGui")
local RunService = game:GetService("RunService")
 
-- Revisar si ya existe el HUD
local existingHUD = gui:FindFirstChild("FPSHUD")
if existingHUD then
    existingHUD:Destroy()
    return
end
 
-- Crear sonido de notificación
local sound = gui:FindFirstChild("FPSNotifySound")
if not sound then
    sound = Instance.new("Sound")
    sound.Name = "FPSNotifySound"
    sound.SoundId = "rbxassetid://911882330" -- Sonido de notificación
    sound.Volume = 1
    sound.Parent = gui
end
sound:Play()
 
-- Crear ScreenGui y TextLabel para HUD
local screenGui = Instance.new("ScreenGui")
screenGui.Name = "FPSHUD"
screenGui.Parent = gui
 
local fpsLabel = Instance.new("TextLabel")
fpsLabel.Size = UDim2.new(0,200,0,50)
fpsLabel.Position = UDim2.new(0,10,0,10)
fpsLabel.BackgroundTransparency = 0.5
fpsLabel.BackgroundColor3 = Color3.fromRGB(121,156,208) -- #799CD0
fpsLabel.TextColor3 = Color3.fromRGB(255,255,255)
fpsLabel.TextScaled = true
fpsLabel.Font = Enum.Font.SourceSansBold
fpsLabel.Text = "FPS: 0 | Ping: 0ms"
fpsLabel.Parent = screenGui
 
-- Variables
local fps = 0
local ping = 0
local frames = 0
local lastTime = tick()
 
-- Calcular FPS
RunService.RenderStepped:Connect(function()
    frames = frames + 1
    if tick() - lastTime >= 1 then
        fps = frames
        frames = 0
        lastTime = tick()
    end
end)
 
-- Calcular ping
RunService.Heartbeat:Connect(function(delta)
    ping = math.floor(delta * 1000) -- ms aproximado
end)
 
-- Actualizar HUD cada segundo
task.spawn(function()
    while fpsLabel.Parent do
        fpsLabel.Text = "FPS: "..fps.." | Ping: "..ping.."ms"
        task.wait(1)
    end
end)
 
   end,
})
 
        local Button = Misc:CreateButton({
   Name = "Check Health",
   Callback = function()
           local player = game.Players.LocalPlayer
local character = player.Character or player.CharacterAdded:Wait()
local humanoid = character:WaitForChild("Humanoid")
local gui = player:WaitForChild("PlayerGui")
 
-- Crear sonido en PlayerGui si no existe
local sound = gui:FindFirstChild("HealthNotifySound")
if not sound then
    sound = Instance.new("Sound")
    sound.Name = "HealthNotifySound"
    sound.SoundId = "rbxassetid://911882330" -- Sonido de notificación
    sound.Volume = 1
    sound.Parent = gui
end
sound:Play()
 
-- Mostrar notificación con salud
pcall(function()
    game.StarterGui:SetCore("SendNotification", {
        Title = "Health Check",
        Text = "Tu salud actual es: " .. math.floor(humanoid.Health) .. " / " .. humanoid.MaxHealth,
        Duration = 5
    })
end)
 
   end,
})
local Button = Misc:CreateButton({
   Name = "Esp Npc Tools Models etc...",
   Callback = function()
   -- ESP para Tools y Models con ProximityPrompt
local Players = game:GetService("Players")
local RunService = game:GetService("RunService")
local Workspace = game:GetService("Workspace")
local StarterGui = game:GetService("StarterGui")
 
local ESPEnabled = true -- toggle para activar/desactivar
 
-- Función para crear highlight en un item
local function highlightItem(item)
    if item:IsA("BasePart") or item:IsA("Model") then
        if item:IsA("Model") then
            if not item.PrimaryPart then
                item.PrimaryPart = item:FindFirstChildWhichIsA("BasePart")
            end
            item = item.PrimaryPart
        end
        if not item:FindFirstChild("ESP_Highlight") then
            local highlight = Instance.new("Highlight")
            highlight.Name = "ESP_Highlight"
            highlight.Adornee = item
            highlight.FillColor = Color3.fromRGB(121, 156, 208) -- azul #799CD0
            highlight.OutlineColor = Color3.fromRGB(255, 255, 255)
            highlight.Parent = item
        end
    end
end
 
-- Loop de ESP
RunService.RenderStepped:Connect(function()
    if ESPEnabled then
        for _, obj in pairs(Workspace:GetDescendants()) do
            if obj:IsA("Tool") then
                highlightItem(obj.Handle or obj)
            elseif obj:IsA("Model") and obj:FindFirstChildWhichIsA("ProximityPrompt") then
                highlightItem(obj)
            end
        end
    else
        -- Limpiar ESP si se desactiva
        for _, obj in pairs(Workspace:GetDescendants()) do
            local h = obj:FindFirstChild("ESP_Highlight")
            if h then h:Destroy() end
        end
    end
end)
 
-- Notificación al ejecutar
StarterGui:SetCore("SendNotification", {
    Title = "ESP Activado",
    Text = "Resaltando Tools y Models con ProximityPrompt",
    Duration = 3
})
 
   end,
})
 
        local Button = Misc:CreateButton({
   Name = "Restore Live",
   Callback = function()
           -- Heal Once Script (para tu simulador de exploits)
-- Solo cura al jugador una vez al ejecutarlo
 
local player = game.Players.LocalPlayer
 
local function healOnce()
    local char = player.Character
    if char then
        local hum = char:FindFirstChildOfClass("Humanoid")
        if hum and hum.MaxHealth > 0 then
            hum.Health = hum.MaxHealth
        end
    end
end
 
-- Ejecutar la curación inmediata
healOnce()
 
   end,
})
 
        local Button = Misc:CreateButton({
   Name = "Invisiblity v2 (z toggle)",
   Callback = function()
           --// Script: Invisibility Toggle con tecla Z
-- Tecla: Z
 
local player = game.Players.LocalPlayer
local userInput = game:GetService("UserInputService")
local sound = Instance.new("Sound", player:WaitForChild("PlayerGui"))
sound.SoundId = "rbxassetid://942127495"
sound.Volume = 1
 
-- Estado global para mantener toggle aunque re-ejecutes
if _G.invis_on == nil then
    _G.invis_on = false
end
 
local function setTransparency(character, transparency)
    for _, part in pairs(character:GetDescendants()) do
        if part:IsA("BasePart") or part:IsA("Decal") then
            part.Transparency = transparency
        end
    end
end
 
local function toggleInvisibility()
    _G.invis_on = not _G.invis_on
    sound:Play()
 
    if _G.invis_on then
        local savedpos = player.Character.HumanoidRootPart.CFrame
        task.wait()
        player.Character:MoveTo(Vector3.new(-25.95, 84, 3537.55))
        task.wait(0.15)
 
        local Seat = Instance.new("Seat", workspace)
        Seat.Anchored = false
        Seat.CanCollide = false
        Seat.Name = "invischair"
        Seat.Transparency = 1
        Seat.Position = Vector3.new(-25.95, 84, 3537.55)
 
        local Weld = Instance.new("Weld", Seat)
        Weld.Part0 = Seat
        Weld.Part1 = player.Character:FindFirstChild("Torso") or player.Character:FindFirstChild("UpperTorso")
 
        task.wait()
        Seat.CFrame = savedpos
        setTransparency(player.Character, 0.5)
 
        game.StarterGui:SetCore("SendNotification", {
            Title = "Invis ON",
            Duration = 3,
            Text = "Eres semi invisible"
        })
    else
        local invisChair = workspace:FindFirstChild("invischair")
        if invisChair then invisChair:Destroy() end
        setTransparency(player.Character, 0)
 
        game.StarterGui:SetCore("SendNotification", {
            Title = "Invis OFF",
            Duration = 3,
            Text = "Ya no eres invisible"
        })
    end
end
 
-- Toggle con tecla Z
userInput.InputBegan:Connect(function(input, gpe)
    if not gpe and input.KeyCode == Enum.KeyCode.Z then
        toggleInvisibility()
    end
end)
 
-- También toggle inmediato al ejecutar
toggleInvisibility()
 
   end,
})

local Button = Misc:CreateButton({
   Name = "Bring all tools and models with PP",
   Callback = function()
           -- 🔹 Script: Traer todos los Tools y Modelos con Handle + ProximityPrompt hacia el jugador
local player = game.Players.LocalPlayer
local character = player.Character or player.CharacterAdded:Wait()
local root = character:WaitForChild("HumanoidRootPart")

-- Función para traer objetos
local function bringObject(obj)
	if obj:IsA("Tool") or (obj:IsA("Model") and obj:FindFirstChild("Handle")) then
		obj:PivotTo(root.CFrame + Vector3.new(0, 5, 0)) -- Los trae justo arriba tuyo
	elseif obj:IsA("BasePart") and obj:FindFirstChildOfClass("ProximityPrompt") then
		obj.CFrame = root.CFrame + Vector3.new(0, 5, 0)
	end
end

-- Buscar en Workspace
for _, obj in pairs(workspace:GetDescendants()) do
	bringObject(obj)
end

-- 🔁 Auto detectar nuevos objetos
workspace.DescendantAdded:Connect(function(obj)
	task.wait(0.2)
	bringObject(obj)
end)

-- 🔊 Notificación y Sonido
game.StarterGui:SetCore("SendNotification", {
	Title = "Bring Items",
	Text = "All Tools & Prompt Objects have been teleported!",
	Duration = 5
})

local sound = Instance.new("Sound")
sound.SoundId = "rbxassetid://12222225" -- Sonido simple
sound.Volume = 3
sound.Parent = root
sound:Play()
   end,
})
        
        local Button = MainTab:CreateButton({
   Name = "Velocity Boost (C)",
   Callback = function()
           --// Script: Speed Boost Toggle
-- Tecla: C
 
local player = game.Players.LocalPlayer
local userInput = game:GetService("UserInputService")
local sound = Instance.new("Sound", player:WaitForChild("PlayerGui"))
sound.SoundId = "rbxassetid://942127495"
sound.Volume = 1
 
local defaultSpeed = 16
local boostedSpeed = 48
local isSpeedBoosted = false
 
local function toggleSpeedBoost()
    isSpeedBoosted = not isSpeedBoosted
    sound:Play()
 
    local humanoid = player.Character and player.Character:FindFirstChild("Humanoid")
    if humanoid then
        if isSpeedBoosted then
            humanoid.WalkSpeed = boostedSpeed
            game.StarterGui:SetCore("SendNotification", {
                Title = "Speed ON",
                Duration = 3,
                Text = "Velocidad: " .. boostedSpeed
            })
        else
            humanoid.WalkSpeed = defaultSpeed
            game.StarterGui:SetCore("SendNotification", {
                Title = "Speed OFF",
                Duration = 3,
                Text = "Velocidad: " .. defaultSpeed
            })
        end
    end
end
 
userInput.InputBegan:Connect(function(input, gpe)
    if not gpe and input.KeyCode == Enum.KeyCode.C then
        toggleSpeedBoost()
    end
end)
 
-- Reset al respawn
player.CharacterAdded:Connect(function(character)
    isSpeedBoosted = false
    local humanoid = character:WaitForChild("Humanoid")
    humanoid.WalkSpeed = defaultSpeed
end)
 
   end,
})
 
        local Trolling = Window:CreateTab("Trolling", 4483362458) -- Title, Image
        local Section = Trolling:CreateSection("Random stuff")
 
        local Button = Trolling:CreateButton({
   Name = "Kill All (need sword)",
   Callback = function()
           -- LocalScript: Kill All en loop sin GUI
local player = game.Players.LocalPlayer
local char = player.Character or player.CharacterAdded:Wait()
 
-- Toggle global
if _G.killAllActive == nil then
    _G.killAllActive = false
end
 
-- Función principal para matar a todos usando firetouchinterest
local function killAllLoop()
    while _G.killAllActive do
        local tool = char:FindFirstChildOfClass("Tool")
        if tool and tool:FindFirstChild("Handle") then
            tool:Activate()
            for _, plr in pairs(game.Players:GetPlayers()) do
                if plr ~= player and plr.Character then
                    local character = plr.Character
                    for _, part in ipairs(character:GetDescendants()) do
                        if part:IsA("BasePart") then
                            firetouchinterest(tool.Handle, part, 0)
                            firetouchinterest(tool.Handle, part, 1)
                        end
                    end
                end
            end
        end
        task.wait(0.5) -- espera medio segundo antes de repetir
    end
end
 
-- Toggle al ejecutar
if not _G.killAllActive then
    _G.killAllActive = true
    spawn(killAllLoop) -- ejecuta en loop en un hilo separado
else
    _G.killAllActive = false
    warn("Kill All loop detenido.")
end
 
   end,
})
 
        local Button = Trolling:CreateButton({
   Name = "Fe Kill npc Katana",
   Callback = function()
           loadstring(game:HttpGet("https://pastebin.com/raw/NTNHKeMZ", true))()
   end,
})
 
        local Button = Trolling:CreateButton({
   Name = "Super Ring",
   Callback = function()
           loadstring(game:HttpGet("https://pastebin.com/raw/aZjaAr6F"))()
   end,
})

local Button = Trolling:CreateButton({
   Name = "Forsaken animations",
   Callback = function()
   		loadstring(game:HttpGet("https://api.luarmor.net/files/v3/loaders/d1226ca454068d9251732f34884a375e.lua"))()
   end,
})

-- 📌 Create Games Tab
local Games = Window:CreateTab("Games", 4483362458)

-- Get current game PlaceId
local placeId = game.PlaceId

-- Supported games list
local supportedGames = {
    [987654321] = "Guts & Blackpowder",
    [13943975949] = "SharkBite 2",
    [123456789] = "Break in 2", -- example placeholder
    [71895508397153] = "Die Of Death", -- replace with real PlaceId
    [81517697608629] = "Pwned by 14:00", -- replace with real PlaceId
    [3851622790] = "Break In", -- replace with real PlaceId
    [79546208627805] = "99 Nights in the Forest",
    [99630341423433] = "Biast" -- replace with real PlaceId
}

-- 🟦 99 Nights in the Forest
    if placeId == 79546208627805 then
    local ForestTab = Games:CreateSection("99 Nights in the Forest")
    local Button = Games:CreateButton({
   Name = " See all hubs (not mine)",
   Callback = function()
   loadstring(game:HttpGet("https://pastebin.com/raw/GreLQtfN"))()
   end,
})

local Button = Games:CreateButton({
   Name = "Gem farm",
   Callback = function()
   getgenv().WebhookURL = "" --If not used do not write anything
loadstring(game:HttpGet("https://raw.githubusercontent.com/caomod2077/Script/refs/heads/main/Farm%20Diamond%20v2.lua"))()
   end,
})

-- 🟦 Guts & Blackpowder
elseif placeId == 987654321 then
    local Section = Games:CreateSection("Guts & Blackpowder")

    Games:CreateButton({
        Name = "Kill aura (K toggle)",
        Callback = function()
            local SetLag = 9.5
SetLag = math.clamp(SetLag, 0, 10)
local WaitTime = SetLag * 0.4
if SetLag > 10 then WaitTime = 1.5 elseif SetLag <= 1 then WaitTime = 0 end

if _G.CombatSystemRunning then return end
_G.CombatSystemRunning = true

local UserInputService = game:GetService("UserInputService")
local player = game:GetService("Players").LocalPlayer
local RunService = game:GetService("RunService")

local ATTACK_COOLDOWN = 1
local lastAttackTime = 0
local currentMode = 2 -- 1: Stop, 2: Normal, 3: Clear Zombie
local ZOMBIE_TYPES = {"Agent", "Slim"}

local osClock = os.clock
local mathRandom = math.random
local Vector3New = Vector3.new
local ipairs = ipairs

local KillAuraEnabled = false

-- Toggle Kill Aura with K
UserInputService.InputBegan:Connect(function(input, gameProcessed)
    if gameProcessed then return end
    if input.KeyCode == Enum.KeyCode.K then
        KillAuraEnabled = not KillAuraEnabled
        warn("🔁 Kill Aura is now: " .. (KillAuraEnabled and "ENABLED" or "DISABLED"))
    end
end)

local function executeAttack()
    if currentMode == 1 then return end
    local currentTime = osClock()
    if currentTime - lastAttackTime < ATTACK_COOLDOWN then return end

    local character = player.Character
    if not character then return end

    local rootPart = character:FindFirstChild("HumanoidRootPart")
    local weapon = character:FindFirstChildWhichIsA("Tool")
    if not (rootPart and weapon) then return end

    local remoteEvent = weapon:FindFirstChildWhichIsA("RemoteEvent")
    if not remoteEvent then
        local meleeBase = weapon:FindFirstChild("MeleeBase")
        if meleeBase then
            remoteEvent = meleeBase:FindFirstChildWhichIsA("RemoteEvent")
        end
    end
    if not remoteEvent then return end

    local rootPos = rootPart.Position
    for _, agent in ipairs(workspace:GetDescendants()) do
        if table.find(ZOMBIE_TYPES, agent.Name) then
            local head = agent:FindFirstChild("Head")
            local humanoid = agent:FindFirstChildOfClass("Humanoid")
            if head and humanoid and humanoid.Health > 0 then
                local dist = (head.Position - rootPos).Magnitude
                if dist <= 19 then
                    local hitPos = head.Position
                    local dir = (hitPos - rootPos).Unit
                    local knockback = dir * 15

                    remoteEvent:FireServer("Swing", "Thrust")
                    remoteEvent:FireServer("HitZombie", agent, hitPos, true, knockback, "Head", Vector3New(mathRandom(), mathRandom(), mathRandom()).Unit)

                    if currentMode == 3 then
                        for i = 1, 4 do
                            task.wait(0.05)
                            remoteEvent:FireServer("Swing", "Thrust")
                            remoteEvent:FireServer("HitZombie", agent, hitPos + Vector3New(0, 0.2 * i, 0), true, knockback * (1 + i * 0.1), "Head", Vector3New(mathRandom(), mathRandom(), mathRandom()).Unit * (1 + i * 0.2))
                        end
                    end
                end
            end
        end
    end

    lastAttackTime = currentTime
end

RunService.Heartbeat:Connect(function()
    task.wait(WaitTime)
    if KillAuraEnabled then
        pcall(executeAttack)
    end
end)

     end,
            
})

    Games:CreateButton({
        Name = "Zombie ESP (T toggle)",
        Callback = function()
            -- ESP Highlight Script (No GUI) with T Toggle

local SetLag = 9.5
SetLag = math.clamp(SetLag, 0, 10)
local WaitTime = SetLag * 0.4
if SetLag > 10 then WaitTime = 1.5 elseif SetLag <= 1 then WaitTime = 0 end

local player = game:GetService("Players").LocalPlayer
local RunService = game:GetService("RunService")
local UserInputService = game:GetService("UserInputService")

local cameraFolder = workspace:WaitForChild("Camera")
local zombieModelName = "m_Zombie"
local ESPEnabled = false

-- Toggle ESP with T
UserInputService.InputBegan:Connect(function(input, gameProcessed)
    if gameProcessed then return end
    if input.KeyCode == Enum.KeyCode.T then
        ESPEnabled = not ESPEnabled
        warn("🔁 ESP is now: " .. (ESPEnabled and "ENABLED" or "DISABLED"))
    end
end)

local highlightConfig = {
    Torch = {Color = Color3.fromRGB(100, 255, 100), Transparency = 0.2},
    Axe = {Color = Color3.fromRGB(255, 100, 100), Transparency = 0.2},
    Default = {Color = Color3.fromRGB(240, 240, 240), Transparency = 0.2}
}

local function applyHighlight(model)
    if not model.PrimaryPart then
        model.PrimaryPart = model:FindFirstChildWhichIsA("BasePart") or model:FindFirstChild("HumanoidRootPart")
        if not model.PrimaryPart then return end
    end

    local camera = workspace.CurrentCamera
    local screenPoint = camera:WorldToViewportPoint(model.PrimaryPart.Position)
    if screenPoint.Z <= 0 then return end

    for _, v in ipairs(model:GetDescendants()) do
        if v:IsA("Highlight") and v.Name:sub(1, 4) == "ZHL_" then
            v:Destroy()
        end
    end

    local highlight = Instance.new("Highlight")
    highlight.Name = "ZHL_" .. tostring(math.random(1, 9999))
    highlight.Adornee = model
    highlight.FillTransparency = highlightConfig.Default.Transparency
    highlight.DepthMode = Enum.HighlightDepthMode.AlwaysOnTop

    if model:FindFirstChild("Torch", true) then
        highlight.FillColor = highlightConfig.Torch.Color
    elseif model:FindFirstChild("Axe", true) then
        highlight.FillColor = highlightConfig.Axe.Color
    else
        highlight.FillColor = highlightConfig.Default.Color
    end

    highlight.Parent = model
end

local function updateHighlights()
    local char = player.Character
    if not char or not char:FindFirstChild("HumanoidRootPart") then return end

    for _, model in ipairs(cameraFolder:GetDescendants()) do
        if model:IsA("Model") and model.Name == zombieModelName and model.PrimaryPart then
            local distance = (char.HumanoidRootPart.Position - model.PrimaryPart.Position).Magnitude
            if distance < 80 then
                applyHighlight(model)
            end
        end
    end
end

RunService.Heartbeat:Connect(function()
    task.wait(WaitTime)
    if ESPEnabled then
        pcall(updateHighlights)
    end
end)
        end,
    })

-- 🟦 SharkBite 2
elseif placeId == 13943975949 then
    local SharkTab = Games:CreateSection("SharkBite 2")

    local player = game.Players.LocalPlayer
    local Players = game:GetService("Players")
    local RunService = game:GetService("RunService")
    local ReplicatedStorage = game:GetService("ReplicatedStorage")
    local character = player.Character or player.CharacterAdded:Wait()
    local backpack = player:WaitForChild("Backpack")
    local hrp = character:WaitForChild("HumanoidRootPart")

    local remote = nil
    local sharkName = nil
    local remotesFolder = ReplicatedStorage:WaitForChild("Projectiles"):WaitForChild("Events"):WaitForChild("Weapons"):WaitForChild("remotes")

    local function hookRemote(remoteObject)
        local mt = getrawmetatable(game)
        setreadonly(mt, false)
        local old = mt.__namecall
        mt.__namecall = newcclosure(function(self, ...)
            local method = getnamecallmethod()
            if self == remoteObject and (method == "FireServer" or method == "InvokeServer") then
                remote = self.Name
            end
            return old(self, ...)
        end)
    end

    for _, r in pairs(remotesFolder:GetChildren()) do
        if r:IsA("RemoteEvent") or r:IsA("RemoteFunction") then
            hookRemote(r)
        end
    end

    remotesFolder.ChildAdded:Connect(function(child)
        if child:IsA("RemoteEvent") or child:IsA("RemoteFunction") then
            hookRemote(child)
        end
    end)

    local function getShark()
        local folder = workspace:FindFirstChild("Sharks")
        if folder then
            for _, shark in pairs(folder:GetChildren()) do
                if shark:IsA("Model") then
                    sharkName = shark.Name
                    return
                end
            end
        end
        sharkName = nil
    end

    local function KillShark()
        for _, tool in pairs(backpack:GetChildren()) do
            if tool:IsA("Tool") then
                tool.Parent = character
            end
        end

        for i = 1, 40 do
            if sharkName and remote then
                local shark = workspace:WaitForChild("Sharks"):FindFirstChild(sharkName)
                local remoteInstance = remotesFolder:FindFirstChild(remote)
                if remoteInstance then
                    remoteInstance:FireServer(shark)
                end
            end
            task.wait()
        end
    end

    local function updateHRP()
        character = player.Character or player.CharacterAdded:Wait()
        backpack = player:WaitForChild("Backpack")
        hrp = character:WaitForChild("HumanoidRootPart")
    end
    player.CharacterAdded:Connect(updateHRP)

    local AutoFarmConn
    SharkTab:CreateToggle({
        Name = "Autofarm (Shoot shark once)",
        CurrentValue = false,
        Callback = function(val)
            if val then
                AutoFarmConn = RunService.Heartbeat:Connect(function(dt)
                    if player.Team and player.Team.Name ~= "Shark" and hrp then
                        if not remote then return end
                        KillShark()
                        getShark()

                        if not sharkName then
                            local chinook = workspace:FindFirstChild("Chinook")
                            if chinook and chinook:FindFirstChild("interior") then
                                hrp.CFrame = chinook.interior.CFrame
                            end
                        else
                            local radius = 1000
                            local speed = math.pi * 2
                            local center = Vector3.new(0, -100, 0)
                            local angle = tick() % (math.pi * 2)
                            local offset = Vector3.new(
                                math.cos(angle) * radius,
                                0,
                                math.sin(angle) * radius
                            )
                            hrp.CFrame = CFrame.new(center + offset, center)
                        end
                    end
                end)
            else
                if AutoFarmConn then
                    AutoFarmConn:Disconnect()
                    AutoFarmConn = nil
                end
            end
        end
    })

    local AutoWinConn
    local function getNearestSurvivor()
        local closest, dist = nil, math.huge
        for _, p in pairs(Players:GetPlayers()) do
            if p ~= player and p.Team and p.Team.Name == "Survivor" then
                local c = p.Character
                local h = c and c:FindFirstChild("Humanoid")
                local pHrp = c and c:FindFirstChild("HumanoidRootPart")
                if h and h.Health > 0 and pHrp then
                    local d = (pHrp.Position - hrp.Position).Magnitude
                    if d < dist then
                        dist = d
                        closest = p
                    end
                end
            end
        end
        return closest
    end

    SharkTab:CreateToggle({
        Name = "Auto Win (as Shark)",
        CurrentValue = false,
        Callback = function(val)
            if val then
                AutoWinConn = RunService.Heartbeat:Connect(function()
                    if player.Team and player.Team.Name == "Shark" then
                        local target = getNearestSurvivor()
                        if target and target.Character and target.Character:FindFirstChild("HumanoidRootPart") then
                            local pos = target.Character.HumanoidRootPart.Position
                            for _, s in pairs(workspace:WaitForChild("Sharks"):GetChildren()) do
                                for _, p in pairs(s:GetDescendants()) do
                                    if p:IsA("BasePart") and p.Name == "Ball" then
                                        p.CFrame = CFrame.new(pos)
                                    end
                                end
                            end
                        end
                    end
                end)
            else
                if AutoWinConn then
                    AutoWinConn:Disconnect()
                    AutoWinConn = nil
                end
            end
        end
    })

    local sharkESPConn = {}
    local function toggleSharkESP(enabled)
        local folder = workspace:FindFirstChild("Sharks")
        if not folder then return end
        for _, c in ipairs(sharkESPConn) do c:Disconnect() end
        table.clear(sharkESPConn)
        for _, d in pairs(folder:GetDescendants()) do
            if d:IsA("Highlight") then
                d.Enabled = enabled
                d.OutlineTransparency = enabled and 0 or 1
                table.insert(sharkESPConn, d:GetPropertyChangedSignal("Enabled"):Connect(function()
                    d.Enabled = enabled
                end))
                table.insert(sharkESPConn, d:GetPropertyChangedSignal("OutlineTransparency"):Connect(function()
                    d.OutlineTransparency = enabled and 0 or 1
                end))
            end
        end
    end

    SharkTab:CreateToggle({
        Name = "Highlight Sharks (ESP)",
        CurrentValue = false,
        Callback = function(val)
            toggleSharkESP(val)
        end
    })

    SharkTab:CreateParagraph({
        Title = "Credits",
        Content = "Script by mafuyu (Main)\nSimple Farm by spam77735"
    })
end

-- 🟦 Break In 2
elseif placeId == 123456789 then
    local Section = Games:CreateSection("Break In 2")
    Games:CreateButton({
        Name = "BreaKing Blitz (very OP)",
        Callback = function()
            
loadstring(game:HttpGet("https://raw.githubusercontent.com/RScriptz/RobloxScripts/main/BreakIn2.lua"))()
        end,
    })

-- 🟦 Break In
elseif placeId == 3851622790 then
    local Section = Games:CreateSection("Break In")
    Games:CreateButton({
        Name = "Break In (very OP)",
        Callback = function()
            loadstring(game:HttpGet("https://raw.githubusercontent.com/TrixAde/Proxima-Hub/main/Main.lua"))()

        end,
    })

-- 🟦 Die of Death
elseif placeId == 71895508397153 then
    local Section = Games:CreateSection("Die Of Death")
    Games:CreateButton({
        Name = "DoD Nexer Hub",
        Callback = function()
            loadstring(game:HttpGet("https://raw.githubusercontent.com/Pro666Pro/Die-Of-Death/refs/heads/main/txt.txt"))()
        end,
    })

        local Players = game:GetService("Players")
local RunService = game:GetService("RunService")
local UserInputService = game:GetService("UserInputService")

local localPlayer = Players.LocalPlayer
local camera = workspace.CurrentCamera

local targetMinMaxHealth = 200
local aimbotEnabled = false
local aimSmoothness = 0.2 -- 0 = instant snap, higher = slower smooth

-- Find closest valid target player (max health >= 200)
local function findClosestTarget()
    local closestPlayer = nil
    local shortestDistance = math.huge
    local localChar = localPlayer.Character
    if not localChar then return nil end
    local localHRP = localChar:FindFirstChild("HumanoidRootPart")
    if not localHRP then return nil end

    for _, player in pairs(Players:GetPlayers()) do
        if player ~= localPlayer and player.Character and player.Character:FindFirstChild("Humanoid") then
            local humanoid = player.Character.Humanoid
            if humanoid.MaxHealth >= targetMinMaxHealth and humanoid.Health > 0 then
                local targetPart = player.Character:FindFirstChild("Head") or player.Character:FindFirstChild("HumanoidRootPart")
                if targetPart then
                    local distance = (targetPart.Position - localHRP.Position).Magnitude
                    if distance < shortestDistance then
                        shortestDistance = distance
                        closestPlayer = player
                    end
                end
            end
        end
    end
    return closestPlayer
end

-- Smoothly aim camera at target position
local function aimAtPosition(targetPos)
    local currentCFrame = camera.CFrame
    local lookVector = (targetPos - currentCFrame.Position).Unit
    local targetCFrame = CFrame.new(currentCFrame.Position, currentCFrame.Position + lookVector)
    camera.CFrame = currentCFrame:Lerp(targetCFrame, aimSmoothness)
end

-- Main aimbot loop
RunService.RenderStepped:Connect(function()
    if aimbotEnabled then
        local targetPlayer = findClosestTarget()
        if targetPlayer and targetPlayer.Character then
            local targetPart = targetPlayer.Character:FindFirstChild("Head") or targetPlayer.Character:FindFirstChild("HumanoidRootPart")
            if targetPart then
                aimAtPosition(targetPart.Position)
            end
        end
    end
end)

-- Rayfield toggle integration example:

local Toggle = Games:CreateToggle({
    Name = "Aimbot Killer",
    CurrentValue = false,
    Flag = "Aimbot200HP",
    Callback = function(value)
        aimbotEnabled = value
    end,
})


-- 🟦 Pwned by 14:00
elseif placeId == 81517697608629 then
    local Section = Games:CreateSection("Pwned by 14:00")
    Games:CreateButton({
        Name = "Inf Stamina",
        Callback = function()
            loadstring(game:HttpGet("https://raw.githubusercontent.com/Minnngocc/PWNED-by-14-00/refs/heads/main/Infinite%20stamina"))()
        end,
    })

-- 🟦 Biast
elseif placeId == 99630341423433 then
    local Section = Games:CreateSection("Biast")

    -- Dropdown example (your item teleporter stuff)
    local player = game.Players.LocalPlayer
    local selectedItem = nil
    local itemsFolder = workspace:FindFirstChild("Items")

    local function scanItems()
        local items = {}
        if itemsFolder then
            for _, obj in pairs(itemsFolder:GetChildren()) do
                if obj:IsA("Model") or obj:IsA("BasePart") then
                    table.insert(items, obj.Name)
                end
            end
        end
        return items
    end

    local Dropdown = Games:CreateDropdown({
        Name = "Select Item",
        Options = scanItems(),
        CurrentOption = {},
        MultipleOptions = false,
        Callback = function(option)
            selectedItem = option[1]
        end,
    })

    Games:CreateButton({
        Name = "Teleport to Item",
        Callback = function()
            if not selectedItem or not itemsFolder then return end
            local obj = itemsFolder:FindFirstChild(selectedItem)
            local hrp = player.Character and player.Character:FindFirstChild("HumanoidRootPart")
            if obj and hrp then
                if obj:IsA("Model") and obj.PrimaryPart then
                    hrp.CFrame = obj.PrimaryPart.CFrame + Vector3.new(0,3,0)
                elseif obj:IsA("BasePart") then
                    hrp.CFrame = obj.CFrame + Vector3.new(0,3,0)
                end
            end
        end,
    })

    Games:CreateButton({
        Name = "Refresh Item List",
        Callback = function()
            Dropdown:Refresh(scanItems())
        end,
    })

    Games:CreateButton({
        Name = "Inf Stamina",
        Callback = function()
            local char = player.Character or player.CharacterAdded:Wait()
            local stamina = char:FindFirstChild("Stamina")
            if stamina then
                local noCD = stamina:FindFirstChild("NoCooldown")
                if noCD and noCD:IsA("BoolValue") then
                    noCD.Value = true
                end
            end
        end,
    })

        local player = game.Players.LocalPlayer
local camera = workspace.CurrentCamera

local runAimbot = false
local HEALTH_THRESHOLD = 200
local AIM_UPDATE_INTERVAL = 0.1

-- Returns the closest player with MaxHealth > 200
local function getClosestTarget()
    local closest = nil
    local shortestDistance = math.huge

    for _, otherPlayer in pairs(game.Players:GetPlayers()) do
        if otherPlayer ~= player and otherPlayer.Character then
            local hrp = otherPlayer.Character:FindFirstChild("HumanoidRootPart")
            local hum = otherPlayer.Character:FindFirstChild("Humanoid")
            if hrp and hum and hum.MaxHealth > HEALTH_THRESHOLD then
                local distance = (player.Character.HumanoidRootPart.Position - hrp.Position).Magnitude
                if distance < shortestDistance then
                    shortestDistance = distance
                    closest = hrp
                end
            end
        end
    end

    return closest
end

-- Point camera at the target
local function lookAt(target)
    if not target then return end
    local camPos = camera.CFrame.Position
    local lookVector = (target.Position - camPos).Unit
    camera.CFrame = CFrame.new(camPos, camPos + lookVector)
end

-- Aimbot main loop
local function aimbotLoop()
    while runAimbot do
        local target = getClosestTarget()
        if target then
            lookAt(target)
        end
        task.wait(AIM_UPDATE_INTERVAL)
    end
end

-- ✅ Rayfield Toggle
local AimbotBiast = Games:CreateToggle({
   Name = "Aimbot Killer",
   CurrentValue = false,
   Flag = "AimbotToggle",
   Callback = function(Value)
       runAimbot = Value
       if runAimbot then
           task.spawn(aimbotLoop)
       end
   end,
})

local player = game.Players.LocalPlayer
local character = player.Character or player.CharacterAdded:Wait()
local hrp = character:WaitForChild("HumanoidRootPart")

-- Settings
local itemsFolder = workspace:FindFirstChild("Items") or workspace
local fallbackPosition = Vector3.new(0, 5, 0)
local itemKeywords = {"teapot", "scroll", "drako"}
local deliveryKeywords = {"delivery", "drop", "submit", "turnin", "goal"}

local pickedItems = {}

-- Simulate key press
local function pressKey(key)
    local vim = game:GetService("VirtualInputManager")
    vim:SendKeyEvent(true, key:lower(), false, game)
    task.wait(0.1)
    vim:SendKeyEvent(false, key:lower(), false, game)
end

-- Find delivery zone
local function findDeliveryPoint()
    for _, obj in pairs(workspace:GetDescendants()) do
        if obj:IsA("BasePart") or obj:IsA("Model") then
            local name = obj.Name:lower()
            for _, keyword in pairs(deliveryKeywords) do
                if name:find(keyword) then
                    print("[Delivery Found]:", obj:GetFullName())
                    return obj
                end
            end
        end
    end
    return nil
end

-- Teleport to object
local function teleportTo(obj)
    if not obj then return end
    if obj:IsA("Model") and obj.PrimaryPart then
        hrp.CFrame = obj.PrimaryPart.CFrame + Vector3.new(0, 3, 0)
    elseif obj:IsA("BasePart") then
        hrp.CFrame = obj.CFrame + Vector3.new(0, 3, 0)
    end
end

-- Find nearest valid item
local function findNextItem()
    local closest = nil
    local shortest = math.huge

    for _, obj in pairs(itemsFolder:GetChildren()) do
        if (obj:IsA("Model") or obj:IsA("BasePart")) and not pickedItems[obj] then
            local name = obj.Name:lower()
            for _, keyword in pairs(itemKeywords) do
                if name:find(keyword) then
                    local objPos = (obj:IsA("Model") and obj.PrimaryPart and obj.PrimaryPart.Position) or obj.Position
                    local distance = (hrp.Position - objPos).Magnitude
                    if distance < shortest then
                        shortest = distance
                        closest = obj
                    end
                end
            end
        end
    end

    if closest then
        pickedItems[closest] = true
    end
    return closest
end

-- ✅ Rayfield Toggle
Games:CreateToggle({
    Name = " (BETA) Auto Farm Items",
    CurrentValue = false,
    Flag = "AutoFarmToggle",
    Callback = function(runFarm)
        if runFarm then
            task.spawn(function()
                local deliveryPoint = findDeliveryPoint()
                while runFarm do
                    task.wait(1)

                    character = player.Character or player.CharacterAdded:Wait()
                    hrp = character:FindFirstChild("HumanoidRootPart")
                    if not hrp then continue end

                    local targetItem = findNextItem()
                    if targetItem then
                        teleportTo(targetItem)
                        task.wait(0.5)
                        pressKey("e")
                        task.wait(1)

                        if deliveryPoint then
                            teleportTo(deliveryPoint)
                        else
                            hrp.CFrame = CFrame.new(fallbackPosition)
                        end

                        task.wait(0.5)

                        local tool = player.Backpack:FindFirstChildOfClass("Tool") or character:FindFirstChildOfClass("Tool")
                        if tool then
                            pressKey("q")
                        end
                    end
                end
            end)
        end
    end,
})

        local Players = game:GetService("Players")
local TweenService = game:GetService("TweenService")

local localPlayer = Players.LocalPlayer
local targetMinMaxHealth = 200
local scaleFactor = 5 -- How much bigger hitboxes get (5x size)

-- Parts to expand (commonly hitbox parts)
local hitboxParts = {
    "HumanoidRootPart",
    "Head",
    "Torso",
    "UpperTorso",
    "LowerTorso"
}

-- Store original sizes and original CFrames (to restore size and avoid shifting)
local originalSizes = {}
local originalCFrames = {}

local runExpand = false

-- Tween info for smooth scaling
local tweenInfo = TweenInfo.new(0.3, Enum.EasingStyle.Quad, Enum.EasingDirection.Out)

local function expandPart(part)
    if not part then return end
    if originalSizes[part] == nil then
        originalSizes[part] = part.Size
        originalCFrames[part] = part.CFrame
    end

    local newSize = originalSizes[part] * scaleFactor
    -- Calculate size diff and adjust CFrame to keep part centered properly
    local sizeDiff = (newSize - part.Size) * 0.5
    local newCFrame = part.CFrame * CFrame.new(0, sizeDiff.Y, 0)

    local tweenSize = TweenService:Create(part, tweenInfo, {Size = newSize, CFrame = newCFrame})
    tweenSize:Play()
end

local function restorePart(part)
    if not part then return end
    if originalSizes[part] then
        local tweenRestore = TweenService:Create(part, tweenInfo, {Size = originalSizes[part], CFrame = originalCFrames[part]})
        tweenRestore:Play()
        originalSizes[part] = nil
        originalCFrames[part] = nil
    end
end

local function expandCharacter(character)
    for _, partName in ipairs(hitboxParts) do
        local part = character:FindFirstChild(partName)
        if part and part:IsA("BasePart") then
            expandPart(part)
        end
    end
end

local function restoreCharacter(character)
    for _, partName in ipairs(hitboxParts) do
        local part = character:FindFirstChild(partName)
        if part and part:IsA("BasePart") then
            restorePart(part)
        end
    end
end

-- Main loop to apply expansion on valid players
task.spawn(function()
    while true do
        task.wait(1)
        if runExpand then
            for _, player in pairs(Players:GetPlayers()) do
                if player ~= localPlayer and player.Character and player.Character:FindFirstChild("Humanoid") then
                    local humanoid = player.Character.Humanoid
                    if humanoid.MaxHealth >= targetMinMaxHealth then
                        expandCharacter(player.Character)
                    else
                        restoreCharacter(player.Character)
                    end
                end
            end
        else
            -- Restore all if toggle off
            for _, player in pairs(Players:GetPlayers()) do
                if player.Character then
                    restoreCharacter(player.Character)
                end
            end
        end
    end
end)

-- Handle character respawn for each player
Players.PlayerAdded:Connect(function(player)
    player.CharacterAdded:Connect(function(character)
        task.wait(1)
        if runExpand and player ~= localPlayer then
            local humanoid = character:WaitForChild("Humanoid")
            if humanoid.MaxHealth >= targetMinMaxHealth then
                expandCharacter(character)
            end
        end
    end)
end)

Players.PlayerRemoving:Connect(function(player)
    if player.Character then
        restoreCharacter(player.Character)
    end
end)

-- Now plug into Rayfield toggle:

local Toggle = Games:CreateToggle({
    Name = "Easy Hitbox Expander",
    CurrentValue = false,
    Flag = "UniversalHitboxExpander",
    Callback = function(value)
        runExpand = value
        if not value then
            -- Restore immediately on toggle off
            for _, player in pairs(Players:GetPlayers()) do
                if player.Character then
                    restoreCharacter(player.Character)
                end
            end
        end
    end,
})

-- 🟥 Unsupported Games
else
    local Section = Games:CreateSection("Unsupported Game")

    -- Build a list of supported game names
    local gameNames = {}
    for _, name in pairs(supportedGames) do
        table.insert(gameNames, "- " .. name)
    end

    -- Show message with supported games list
    Games:CreateParagraph({
        Title = "Oops!",
        Content = "Looks like you are in an unsupported game.\n\nSupported Games:\n" ..
            table.concat(gameNames, "\n")
    })
end

        -- Useful stuff down here
        -- TP TO NEAREST TOOL/MODEL WITH HANDLE + PROXIMITYPROMPT
local function tpNearest()
    local player = game.Players.LocalPlayer
    if not player.Character or not player.Character:FindFirstChild("HumanoidRootPart") then return end

    local root = player.Character.HumanoidRootPart
    local nearest, dist = nil, math.huge

    for _, obj in ipairs(workspace:GetDescendants()) do
        if (obj:IsA("Tool") or obj:IsA("Model")) and obj:FindFirstChild("Handle") and obj:FindFirstChildWhichIsA("ProximityPrompt", true) then
            local handle = obj:FindFirstChild("Handle")
            if handle then
                local d = (handle.Position - root.Position).Magnitude
                if d < dist then
                    dist = d
                    nearest = handle
                end
            end
        end
    end

    if nearest then
        root.CFrame = nearest.CFrame + Vector3.new(0,3,0)
    end
end

Misc:CreateButton({
    Name = "TP to Nearest Tool/Model",
    Callback = tpNearest
})
        
       -- Crear Tab
local Useful = Window:CreateTab("Useful", 4483362458) -- Título y icono

-- Sección Protection
local Inmunity = Useful:CreateSection("Protection")

-- Anti Void
Useful:CreateToggle({
    Name = "Anti Void",
    CurrentValue = false,
    Flag = "AntiVoid",
    Callback = function(Value)
        _G.AntiVoid = Value
        if Value then
            spawn(function()
                while _G.AntiVoid do
                    local player = game.Players.LocalPlayer
                    if player.Character and player.Character:FindFirstChild("HumanoidRootPart") then
                        if player.Character.HumanoidRootPart.Position.Y < -20 then
                            player.Character.HumanoidRootPart.CFrame = CFrame.new(0,50,0)
                        end
                    end
                    task.wait(0.2)
                end
            end)
        end
    end
})

-- Anti Seat Lock
Useful:CreateToggle({
    Name = "Anti Seat Lock",
    CurrentValue = false,
    Flag = "AntiSeat",
    Callback = function(Value)
        _G.AntiSeat = Value
        if Value then
            spawn(function()
                while _G.AntiSeat do
                    local player = game.Players.LocalPlayer
                    if player.Character then
                        local hum = player.Character:FindFirstChildOfClass("Humanoid")
                        if hum and hum.Sit then
                            hum.Sit = false
                        end
                    end
                    task.wait(0.2)
                end
            end)
        end
    end
})

-- Anti Freeze
Useful:CreateToggle({
    Name = "Anti Freeze",
    CurrentValue = false,
    Flag = "AntiFreeze",
    Callback = function(Value)
        _G.AntiFreeze = Value
        if Value then
            spawn(function()
                while _G.AntiFreeze do
                    local player = game.Players.LocalPlayer
                    if player.Character and player.Character:FindFirstChild("HumanoidRootPart") then
                        local root = player.Character.HumanoidRootPart
                        if root.Anchored == true then
                            root.Anchored = false
                        end
                    end
                    task.wait(0.2)
                end
            end)
        end
    end
})

-- Reset Stats
Useful:CreateToggle({
    Name = "Reset Stats",
    CurrentValue = false,
    Flag = "ResetStats",
    Callback = function(Value)
        _G.ResetStats = Value
        local player = game.Players.LocalPlayer
        local defaultWalkSpeed = 16
        local defaultJumpPower = 50

        local function applyReset(char)
            local hum = char:WaitForChild("Humanoid")
            hum:GetPropertyChangedSignal("WalkSpeed"):Connect(function()
                if _G.ResetStats and hum.WalkSpeed ~= defaultWalkSpeed then
                    hum.WalkSpeed = defaultWalkSpeed
                end
            end)
            hum:GetPropertyChangedSignal("JumpPower"):Connect(function()
                if _G.ResetStats and hum.JumpPower ~= defaultJumpPower then
                    hum.JumpPower = defaultJumpPower
                end
            end)
        end

        if player.Character then
            applyReset(player.Character)
        end
        player.CharacterAdded:Connect(applyReset)
    end
})

-- Sección Teleport
local TP = Useful:CreateSection("Teleport")

-- Variable global para guardar posición
_G.SavedPosition = nil

-- Botón Set Position
Useful:CreateButton({
    Name = "Set Position",
    Callback = function()
        local player = game.Players.LocalPlayer
        local hrp = player.Character and player.Character:FindFirstChild("HumanoidRootPart")
        if hrp then
            _G.SavedPosition = hrp.Position
            game.StarterGui:SetCore("SendNotification", {
                Title = "Position Saved",
                Text = "Coordinates saved successfully!",
                Duration = 3
            })
        end
    end
})

-- Botón Teleport
Useful:CreateButton({
    Name = "Teleport to Saved Position",
    Callback = function()
        local player = game.Players.LocalPlayer
        local hrp = player.Character and player.Character:FindFirstChild("HumanoidRootPart")
        if hrp then
            if _G.SavedPosition then
                hrp.CFrame = CFrame.new(_G.SavedPosition)
                game.StarterGui:SetCore("SendNotification", {
                    Title = "Teleported",
                    Text = "You have been teleported to the saved position!",
                    Duration = 3
                })
            else
                game.StarterGui:SetCore("SendNotification", {
                    Title = "Error",
                    Text = "No saved position found. Press 'Set Position' first.",
                    Duration = 3
                })
            end
        end
    end
})

local Combat = Useful:CreateSection("Combat")

local Button = Useful:CreateButton({
   Name = "Enable Jason Hitbox",
   Callback = function()
   		loadstring(game:HttpGet("https://raw.githubusercontent.com/Qwoqeex/Hitbox-Expander/refs/heads/main/Hitboxes"))()
   end,
})

    local Fun = Window:CreateTab("Fun", 4483362458) -- Title, Image
    local Section = Fun:CreateSection("Animations (FE)")

    -- 🎭 BluuGui Fun Tab: Ultimate Animation Changer

local Players = game:GetService("Players")
local LocalPlayer = Players.LocalPlayer


-- State
local currentTrack = nil
local loopEnabled = false
local userAnimId = ""
local movementAnims = {}

-- Preset animations
local presetAnims = {
    ["R6 - Dance"] = "182435998",
    ["R6 - Wave"] = "128777973",
    ["R6 - Point"] = "128853357",
    ["R6 - Cheer"] = "129423030",
    ["R6 - Laugh"] = "129423131",

    ["R15 - Dance 1"] = "507771019",
    ["R15 - Dance 2"] = "507776043",
    ["R15 - Dance 3"] = "507777268",
    ["R15 - Wave"] = "507770239",
    ["R15 - Point"] = "507770453",
    ["R15 - Cheer"] = "507770677",
    ["R15 - Laugh"] = "507770818",

    ["Monster Mash"] = "107864975312860",      -- Bluudud’s favorite
    ["Russian Dance"] = "79145094121685",     -- UGC emote :contentReference[oaicite:2]{index=2}
}

-- Animation Playback Helper
local function playAnim(id, name)
    local char = LocalPlayer.Character or LocalPlayer.CharacterAdded:Wait()
    local humanoid = char:FindFirstChildOfClass("Humanoid")
    if not humanoid then return end

    -- Stop same-name track to prevent overlap
    for _, tr in ipairs(humanoid:GetPlayingAnimationTracks()) do
        if tr.Name == name then tr:Stop() end
    end

    local anim = Instance.new("Animation")
    anim.AnimationId = "rbxassetid://" .. id
    local track = humanoid:LoadAnimation(anim)
    track.Name = name
    track.Looped = loopEnabled
    track:Play()

    currentTrack = track
    warn("Playing: " .. name .. " (Loop: " .. tostring(loopEnabled) .. ")")
end

-- UI: Loop Toggle
Fun:CreateToggle({
    Name = "Loop Animation",
    CurrentValue = false,
    Callback = function(val) loopEnabled = val end
})

-- UI: Custom Animation Input
Fun:CreateInput({
    Name = "Custom Animation ID",
    PlaceholderText = "Paste any animation ID",
    RemoveTextAfterFocusLost = false,
    Callback = function(text) userAnimId = text end,
})

Fun:CreateButton({
    Name = "▶ Play Custom Animation",
    Callback = function()
        if userAnimId ~= "" then
            playAnim(userAnimId, "CustomAnim")
        else
            warn(" No animation ID entered!")
        end
    end,
})

-- UI: Preset Animations Dropdown
Fun:CreateDropdown({
    Name = "Preset Animations",
    Options = table.keys(presetAnims),
    CurrentOption = {},
    MultipleOptions = false,
    Callback = function(opt)
        local name = opt[1]
        playAnim(presetAnims[name], name)
    end,
})

-- UI: Stop Animation
Fun:CreateButton({
    Name = "⏹ Stop Animation",
    Callback = function()
        if currentTrack then
            currentTrack:Stop()
            currentTrack = nil
            warn("Animation stopped.")
        else
            warn("No animation playing.")
        end
    end,
})

--  Movement Animations Replacer
local defaultTypes = { "Idle", "Walk", "Run", "Jump", "Swim" }
for _, animType in ipairs(defaultTypes) do
    Fun:CreateInput({
        Name = animType .. " Animation ID",
        PlaceholderText = "Enter animation ID",
        RemoveTextAfterFocusLost = false,
        Callback = function(id)
            movementAnims[animType] = id
            local char = LocalPlayer.Character or LocalPlayer.CharacterAdded:Wait()
            local humanoid = char:WaitForChild("Humanoid")
            local animate = char:FindFirstChild("Animate")

            if animate and id ~= "" then
                local animObj
                if animType == "Idle" then
                    animObj = animate:FindFirstChild("idle") and animate.idle:FindFirstChild("Animation1")
                else
                    animObj = animate:FindFirstChild(animType:lower())
                end

                if animObj and animObj:IsA("Animation") then
                    animObj.AnimationId = "rbxassetid://" .. id
                end
            end
        end,
    })
end

-- 🎵 Fun Section: Music Player

local MusicSection = Fun:CreateSection("Music Player (No FE)")

-- Sound holder
local sound = Instance.new("Sound")
sound.Parent = game:GetService("SoundService")
sound.Looped = true
sound.Volume = 1 -- default volume

-- Premade tracks
local premadeTracks = {
    ["It's Raining Tacos"] = "142376088", -- classic meme song
    ["Bluudanc"] = "9121645091", -- placeholder, replace with your track
    ["Random Meme 1"] = "184735242", -- Darude Sandstorm
    ["Random Meme 2"] = "458726741", -- Crab Rave
}

-- Input for custom music
Fun:CreateInput({
    Name = "Custom SoundId",
    PlaceholderText = "Enter SoundId here",
    RemoveTextAfterFocusLost = false,
    Callback = function(value)
        if tonumber(value) then
            sound.SoundId = "rbxassetid://" .. value
            sound:Play()
        else
            warn("Invalid SoundId")
        end
    end,
})

-- Dropdown for premade tracks
Fun:CreateDropdown({
    Name = "Premade Songs",
    Options = {"It's Raining Tacos", "Bluudanc", "Random Meme 1", "Random Meme 2"},
    CurrentOption = {},
    MultipleOptions = false,
    Callback = function(option)
        local id = premadeTracks[option[1]]
        if id then
            sound.SoundId = "rbxassetid://" .. id
            sound:Play()
        end
    end,
})

-- Buttons
Fun:CreateButton({
    Name = "▶️ Play",
    Callback = function()
        if sound.SoundId ~= "" then
            sound:Play()
        end
    end,
})

Fun:CreateButton({
    Name = "⏹ Stop",
    Callback = function()
        sound:Stop()
    end,
})

-- Volume Slider
Fun:CreateSlider({
    Name = "🔊 Volume",
    Range = {0, 10},
    Increment = 1,
    CurrentValue = 5,
    Callback = function(value)
        sound.Volume = value / 10 -- scale 0–10 slider to 0–1 volume
    end,
})

-- Loop Toggle
Fun:CreateToggle({
    Name = "🔁 Loop Song",
    CurrentValue = true,
    Callback = function(val)
        sound.Looped = val
    end,
})

-- 🎡 SpinBot Section
local SpinSection = Fun:CreateSection("SpinBot")

local spinning = false
local spinConnection
local spinVelocity = 10 -- default spin speed

-- Spin Toggle
Fun:CreateToggle({
    Name = "🌀 Enable SpinBot",
    CurrentValue = false,
    Callback = function(val)
        local player = game.Players.LocalPlayer
        local char = player.Character or player.CharacterAdded:Wait()
        local hrp = char:WaitForChild("HumanoidRootPart")

        if val then
            spinning = true
            spinConnection = game:GetService("RunService").Heartbeat:Connect(function()
                if hrp and spinning then
                    hrp.CFrame = hrp.CFrame * CFrame.Angles(0, math.rad(spinVelocity), 0)
                end
            end)
        else
            spinning = false
            if spinConnection then
                spinConnection:Disconnect()
                spinConnection = nil
            end
        end
    end,
})

-- Velocity Slider
Fun:CreateSlider({
    Name = "⚡ Spin Speed",
    Range = {1, 100},
    Increment = 1,
    CurrentValue = 10,
    Callback = function(value)
        spinVelocity = value
    end,
})

-- 🌀 Fun Tab Extras

-- 🌐 Section
local FunExtras = Fun:CreateSection("Bluu Extras")

-- ⚡ Chat Spammer
local chatSpamConnection
local spamMessages = {
    "💙 BluuGang on top!", 
    "💎 BluuWord forever!", 
    "🔥 Powered by BluuGui!", 
    "🌀 Spin with Bluu!"
}

Fun:CreateToggle({
    Name = "💬 Chat Spammer",
    CurrentValue = false,
    Callback = function(val)
        local ReplicatedStorage = game:GetService("ReplicatedStorage")
        local chatEvent = ReplicatedStorage:WaitForChild("DefaultChatSystemChatEvents"):WaitForChild("SayMessageRequest")

        if val then
            chatSpamConnection = task.spawn(function()
                while task.wait(3) do -- every 3 seconds
                    if not chatSpamConnection then break end
                    local msg = spamMessages[math.random(1, #spamMessages)]
                    chatEvent:FireServer(msg, "All")
                end
            end)
        else
            chatSpamConnection = nil
        end
    end,
})

-- 🐢 Fake Lag
local fakeLag = false
local lagConnection

Fun:CreateToggle({
    Name = "🐢 Fake Lag",
    CurrentValue = false,
    Callback = function(val)
        local RunService = game:GetService("RunService")
        local player = game.Players.LocalPlayer
        local char = player.Character or player.CharacterAdded:Wait()
        local hrp = char:WaitForChild("HumanoidRootPart")

        if val then
            fakeLag = true
            lagConnection = RunService.Heartbeat:Connect(function()
                if fakeLag and hrp then
                    hrp.Anchored = true
                    task.wait(1) -- "freeze" for 1 second
                    hrp.Anchored = false
                end
            end)
        else
            fakeLag = false
            if lagConnection then
                lagConnection:Disconnect()
                lagConnection = nil
            end
        end
    end,
})

-- 🌊 Wow Thingz are geting a whole Bluu (World Recolor)
local bluuEnabled = false
local bluuConnection

Fun:CreateToggle({
    Name = "🌊 Wow Thingz are geting a whole Bluu",
    CurrentValue = false,
    Callback = function(val)
        bluuEnabled = val
        local player = game.Players.LocalPlayer

        if bluuEnabled then
            -- Apply blue filter loop
            bluuConnection = game:GetService("RunService").Heartbeat:Connect(function()
                for _, obj in ipairs(workspace:GetDescendants()) do
                    if obj:IsA("BasePart") and obj:IsDescendantOf(player.Character) == false then
                        obj.Color = Color3.fromRGB(0, 162, 255) -- Roblox Blue tone
                        obj.Material = Enum.Material.SmoothPlastic -- smooth look
                    elseif obj:IsA("Decal") or obj:IsA("Texture") then
                        obj.Color3 = Color3.fromRGB(0, 162, 255)
                    end
                end
            end)
        else
            -- Disable effect and restore (optional reset)
            if bluuConnection then
                bluuConnection:Disconnect()
                bluuConnection = nil
            end
        end
    end,
})

    local ModeratorTab = Window:CreateTab("Advanced Moderator Sistem", 4483362458) -- Title, Image
    local Section = ModeratorTab:CreateSection("Configuration")

    -- =========================================================
-- Moderator Tab (Advanced) by 118o8
-- Requires your UI lib: Window:CreateTab / :CreateToggle / :CreateButton
-- =========================================================

local Players            = game:GetService("Players")
local ReplicatedStorage  = game:GetService("ReplicatedStorage")
local RunService         = game:GetService("RunService")
local StarterGui         = game:GetService("StarterGui")
local LocalPlayer        = Players.LocalPlayer

-- ===== UI Tab =====
local ModeratorTab = Window:CreateTab("Moderator", 4483362458)

-- ===== State =====
local isRunning = false
local toggles = {
    AntiCheat = false,
    AntiToxic = false,
    AntiSpam  = false,
    AntiBot   = false,
    AntiAFK   = false,
    AntiNSFW  = false,
    Logging   = false,
}

-- connection buckets so we can cleanly stop
local conns = {}
local function bind(name, conn)
    conns[name] = conns[name] or {}
    table.insert(conns[name], conn)
end
local function unbindAll()
    for _, list in pairs(conns) do
        for _, c in ipairs(list) do
            if typeof(c) == "RBXScriptConnection" then c:Disconnect() end
        end
    end
    conns = {}
end

-- ===== Utils =====
local function say(msg)
    local ok, chat = pcall(function()
        return ReplicatedStorage:WaitForChild("DefaultChatSystemChatEvents"):WaitForChild("SayMessageRequest")
    end)
    if ok and chat then
        chat:FireServer(msg, "All")
    end
end

local function notify(title, text, dur)
    pcall(function()
        StarterGui:SetCore("SendNotification", {
            Title = title or "Moderator",
            Text = text or "",
            Duration = dur or 3
        })
    end)
end

-- Guard: toggles changed before Execute → just notify (no chat)
local function requireRunning(name)
    if not isRunning then
        notify("Moderator", "'" .. name .. "' requires Execute first!", 4)
        return false
    end
    return true
end

-- ===== Data / Rules =====
local STRIKE_WINDOW = 300 -- seconds to keep strike timestamps relevant (rolling window)
local REPORT_DELAY_AFTER_3RD = 180 -- 3 minutes “recording” window

local playerState = {} -- per-user tracking
local function ensurePlayerState(plr)
    local st = playerState[plr.UserId]
    if st then return st end
    st = {
        lastPos = nil,
        lastMove = os.clock(),
        lastChatTime = 0,
        chatTimes = {},
        repeatedCache = {}, -- msg->count
        strikes = { toxic = {}, spam = {}, cheat = {} },
        lastStrikeAt = { toxic = 0, spam = 0, cheat = 0 },
        flagged3rdAt = { toxic = 0, spam = 0, cheat = 0 },
        nsfwFlagged = false,
        isAFK = false,
        lastJumpAt = 0,
        lastSpeedCheckAt = 0,
        lastTeleportCheckAt = 0,
        lastFlingCheckAt = 0,
        lastNoclipCheckAt = 0,
        lastGodCheckAt = 0,
    }
    playerState[plr.UserId] = st
    return st
end

-- keyword lists
local TOXIC_PATTERNS = {
    "kill yourself", "kys", "idiot", "noob", "ez", "skill issue",
    "retard", "trash", "garbage", "loser", "git gud", "cry more",
    "faggot", "cancer", "get rekt", "reported"
}
local NSFW_PATTERNS = {
    "r63","18%+","18%p","18%","nsfw","nude","naked","sex","boob","thigh","s%xy","lewd"
}
-- normalize patterns to plain find by pre-lowering messages and names. % handles variations.

local function matchAny(msg, list)
    msg = msg:lower()
    for _, kw in ipairs(list) do
        if msg:find(kw) then return true end
    end
    return false
end

-- Simple Roblox report attempt (guarded); may be restricted in normal experiences
local function tryReport(plr, category, details)
    pcall(function()
        -- Not all executors / environments allow this:
        -- Players:ReportAbuse(player, reason, message) is CoreScript-only in many cases.
        -- We pcall to avoid errors.
        Players:ReportAbuse(plr, category or "Bullying", details or "Automated moderator report.")
    end)
end

-- strike handling
local function addStrike(plr, kind, reason)
    local st = ensurePlayerState(plr)
    local now = os.clock()
    local bucket = st.strikes[kind]
    table.insert(bucket, now)

    -- prune old strikes outside window
    local i = 1
    while i <= #bucket do
        if now - bucket[i] > STRIKE_WINDOW then
            table.remove(bucket, i)
        else
            i = i + 1
        end
    end

    local count = #bucket
    if count == 1 then
        say(("[Moderator] %s %s (%s)"):format(plr.Name, kind == "toxic" and "used toxic language" or kind == "spam" and "is spamming" or "triggered cheat checks", reason or ""))
    elseif count == 2 then
        say(("[Moderator] %s second strike for %s. Attempting Roblox report."):format(plr.Name, kind))
        tryReport(plr, kind == "toxic" and "Bullying" or "Scamming", "[Auto] "..(reason or kind))
    elseif count >= 3 then
        say(("[Moderator] %s third strike for %s. Recording & logging…"):format(plr.Name, kind))
        st.flagged3rdAt[kind] = now
        -- schedule the “video recorded” announcement after REPORT_DELAY_AFTER_3RD if still in game
        task.spawn(function(uid, k)
            local pl = Players:GetPlayerByUserId(uid)
            task.wait(REPORT_DELAY_AFTER_3RD)
            if pl and playerState[uid] and playerState[uid].flagged3rdAt[k] > 0 then
                say(("[Moderator] Evidence window complete for %s — video recorded, sending to devs…"):format(pl.Name))
            end
        end, plr.UserId, kind)
    end

    -- optional logging
    if toggles.Logging then
        warn(("[LOG] %s | %s | %s"):format(plr.Name, kind, reason or ""))
    end
end

-- ========= Modules =========

-- Anti-Spam + Toxic (chat listener)
local function startChatWatch()
    -- attach to current & future players
    local function onPlayer(plr)
        local st = ensurePlayerState(plr)
        bind("Chat_"..plr.UserId, plr.Chatted:Connect(function(msg)
            local now = os.clock()

            -- SPAM
            if toggles.AntiSpam then
                -- cooldown
                local tooFast = (now - st.lastChatTime) < 1.8
                st.lastChatTime = now

                -- rolling window
                table.insert(st.chatTimes, now)
                while #st.chatTimes > 0 and now - st.chatTimes[1] > 10 do
                    table.remove(st.chatTimes, 1)
                end

                -- repeated text pattern
                local m = msg:lower()
                st.repeatedCache[m] = (st.repeatedCache[m] or 0) + 1
                -- prune repeats occasionally
                if math.random() < 0.1 then
                    for k, t in pairs(st.repeatedCache) do
                        if t <= 1 then st.repeatedCache[k] = nil end
                    end
                end

                if tooFast or #st.chatTimes >= 6 or st.repeatedCache[m] >= 3 then
                    addStrike(plr, "spam", tooFast and "cooldown<1.8s" or (#st.chatTimes>=6 and "6 msgs in 10s" or "repeated msg x3"))
                end
            end

            -- TOXIC
            if toggles.AntiToxic then
                if matchAny(msg, TOXIC_PATTERNS) then
                    addStrike(plr, "toxic", "keyword match")
                end
            end

            -- BOT hints (chat-based)
            if toggles.AntiBot then
                if #msg > 5 and msg:lower():find("free robux") or msg:lower():find("discord%.gg") then
                    say(("[BotDetector] %s likely a bot (scam link)."):format(plr.Name))
                end
            end

            -- Logging
            if toggles.Logging then
                warn(("[CHAT] %s: %s"):format(plr.Name, msg))
            end
        end))
    end

    for _, p in ipairs(Players:GetPlayers()) do onPlayer(p) end
    bind("Chat_PlayerAdded", Players.PlayerAdded:Connect(onPlayer))
end

-- Anti-Cheat (basic heuristics)
local function startCheatWatch()
    bind("Cheat_Heartbeat", RunService.Heartbeat:Connect(function(dt)
        for _, plr in ipairs(Players:GetPlayers()) do
            if not plr.Character then continue end
            local st = ensurePlayerState(plr)
            local hrp = plr.Character:FindFirstChild("HumanoidRootPart")
            local hum = plr.Character:FindFirstChildOfClass("Humanoid")
            if not hrp or not hum then continue end

            -- teleport (big jump)
            local now = os.clock()
            if st.lastPos then
                local dist = (hrp.Position - st.lastPos).Magnitude
                if dist > 60 and (now - st.lastTeleportCheckAt) > 4 then
                    st.lastTeleportCheckAt = now
                    addStrike(plr, "cheat", "teleport jump "..math.floor(dist))
                end

                -- crude speed check
                local speed = dist / math.max(dt, 1/60)
                if speed > 80 and (now - st.lastSpeedCheckAt) > 4 then
                    st.lastSpeedCheckAt = now
                    addStrike(plr, "cheat", "high speed "..math.floor(speed))
                end
            end

            -- noclip hint
            if hrp.CanCollide == false and (now - st.lastNoclipCheckAt) > 5 then
                st.lastNoclipCheckAt = now
                addStrike(plr, "cheat", "noclip hint")
            end

            -- godmode hint
            if hum.Health > hum.MaxHealth and (now - st.lastGodCheckAt) > 5 then
                st.lastGodCheckAt = now
                addStrike(plr, "cheat", "health>"..math.floor(hum.MaxHealth))
            end

            -- fling hint
            if hrp.AssemblyLinearVelocity.Magnitude > 220 and (now - st.lastFlingCheckAt) > 5 then
                st.lastFlingCheckAt = now
                addStrike(plr, "cheat", "extreme velocity")
            end

            -- AFK tracking base timestamp
            st.lastPos = hrp.Position
        end
    end))
end

-- Anti-AFK (announce players idle > N)
local AFK_TIMEOUT = 300 -- 5 min
local function startAFKWatch()
    bind("AFK_Heartbeat", RunService.Heartbeat:Connect(function()
        for _, plr in ipairs(Players:GetPlayers()) do
            local st = ensurePlayerState(plr)
            local char = plr.Character
            local hrp = char and char:FindFirstChild("HumanoidRootPart")
            if hrp then
                if st.lastPos and (hrp.Position - st.lastPos).Magnitude > 1 then
                    st.lastMove = os.clock()
                    if st.isAFK then
                        st.isAFK = false
                        say(("[AntiAFK] %s is no longer AFK."):format(plr.Name))
                    end
                end
                st.lastPos = hrp.Position

                if not st.isAFK and (os.clock() - st.lastMove) > AFK_TIMEOUT then
                    st.isAFK = true
                    say(("[AntiAFK] %s is AFK for too long."):format(plr.Name))
                end
            end
        end
    end))
end

-- Anti-Bot (basic heuristics)
local function startBotWatch()
    local function checkBot(plr)
        local reasons = {}
        if plr.AccountAge <= 2 then table.insert(reasons, "new account") end
        if plr.DisplayName:len() <= 3 or plr.Name:match("^%l%d%d%d+") then
            table.insert(reasons, "suspicious name")
        end
        if #reasons > 0 then
            say(("[BotDetector] %s flagged: %s"):format(plr.Name, table.concat(reasons, ", ")))
        end
    end
    for _, p in ipairs(Players:GetPlayers()) do checkBot(p) end
    bind("Bot_PlayerAdded", Players.PlayerAdded:Connect(checkBot))
end

-- NSFW Avatar scan (name-based; immediate report)
local function startNSFWWard()
    local function scanChar(plr, character)
        local st = ensurePlayerState(plr)
        if st.nsfwFlagged then return end

        local suspicious = false
        -- check accessories / clothing names
        for _, d in ipairs(character:GetDescendants()) do
            if d:IsA("Accessory") or d:IsA("Shirt") or d:IsA("Pants") or d:IsA("ShirtGraphic") then
                local nm = (d.Name or ""):lower()
                if matchAny(nm, NSFW_PATTERNS) then suspicious = true break end
            end
        end

        if suspicious then
            st.nsfwFlagged = true
            say(("[NSFW] %s avatar flagged. Reporting."):format(plr.Name))
            tryReport(plr, "InappropriateContent", "[Auto] NSFW name/accessory detection.")
        end
    end

    local function onPlayer(plr)
        local function hookChar(char)
            scanChar(plr, char)
            bind("NSFW_"..plr.UserId.."_Added", char.DescendantAdded:Connect(function()
                scanChar(plr, char)
            end))
        end
        if plr.Character then hookChar(plr.Character) end
        bind("NSFW_"..plr.UserId.."_Char", plr.CharacterAdded:Connect(hookChar))
    end

    for _, p in ipairs(Players:GetPlayers()) do onPlayer(p) end
    bind("NSFW_PlayerAdded", Players.PlayerAdded:Connect(onPlayer))
end

-- ========= UI Controls =========

-- Toggles (silent error if not executed)
local order = {"AntiCheat","AntiToxic","AntiSpam","AntiBot","AntiAFK","AntiNSFW","Logging"}
for _, name in ipairs(order) do
    ModeratorTab:CreateToggle({
        Name = name,
        CurrentValue = false,
        Callback = function(on)
            if not isRunning then
                notify("Moderator", "'"..name.."' requires Execute first!", 4)
                return
            end
            toggles[name] = on
            notify("Moderator", name.." = "..tostring(on), 2)
        end
    })
end

-- Execute
ModeratorTab:CreateButton({
    Name = "Execute Moderation",
    Callback = function()
        if isRunning then
            notify("Moderator", "Already running.", 3)
            return
        end
        isRunning = true

        -- Initial chat sequence
        say("Executing...")
        task.wait(3)
        say("Loading...")
        task.wait(2)
        say("✅ Advanced Moderator Script by 118o8")

        -- Start modules (listeners always mount; actions gated by toggles at runtime)
        startChatWatch()
        startCheatWatch()
        startAFKWatch()
        startBotWatch()
        startNSFWWard()

        notify("Moderator", "Running. Flip toggles as needed.", 4)
    end
})

-- Stop
ModeratorTab:CreateButton({
    Name = "Stop Moderation",
    Callback = function()
        if not isRunning then
            notify("Moderator", "Nothing to stop.", 3)
            return
        end
        unbindAll()
        isRunning = false

        -- clear runtime state if you want a fresh start next run
        -- playerState = {} -- (optional) comment this in if you want to forget strikes/logs when stopping

        say("🛑 Moderator Script stopped.")
        notify("Moderator", "Stopped & cleaned up.", 3)
    end
})

-- 🌐 Community Tab
local Community = Window:CreateTab("Community", 4483362458)
local Section = Community:CreateSection("Bluu Updates & News")

-- Permanent Pastebin raw link
local newsLink = "https://pastebin.com/raw/pnU7exSu" -- replace with your real raw link

-- Fetch and show updates
local function loadNews()
    local success, response = pcall(function()
        return game:HttpGet(newsLink)
    end)

    if success and response then
        Community:CreateParagraph({
            Title = "📢 Bluu Updates",
            Content = response
        })
    else
        Community:CreateParagraph({
            Title = "⚠️ Error",
            Content = "Could not load the latest updates. Please check the Pastebin link."
        })
    end
end

-- Load at start
loadNews()

-- Refresh button
Community:CreateButton({
    Name = "🔄 Refresh News",
    Callback = function()
        loadNews()
    end,
})